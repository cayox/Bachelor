\section{Fazit}

Das im Rahmen dieser Arbeit entwickelte Programm ``TestHub'' konnte erfolgreich 
als \gls{MVP} umgesetzt werden. Fast alle der zuvor definierten Anforderungen wurden
erfüllt. Somit wurden auch die zu Beginn angesprochenen Probleme, wie etwa der 
komplizierte Prozess zum Einsehen eines Wartungstermins, verbessert und gelöst.
Dazu wurde ein Joyson Safety Systems spezifisches Programm entwickelt, welches
genau auf das teilweise stark individualisierte Jira-Ticket-System angepasst wurde. 
Nicht nur \gls{Jira} Informationen wurden in TestHub eingebunden, sondern es wurde auch 
eine zentrale und universelle einsetzbare Anlaufstelle für jegliche weitere 
Ressourcen geschaffen. Durch TestHub sind alle Ressourceninformationen, welche
benötigt werden, um einen Test aufzubauen und durchzuführen, an der gleichen Stelle 
und lassen sich schnell und effizient einsehen und bearbeiten. Das in der 
Client-Server Architektur entwickelte System könnte vielen Mitarbeitern
die Arbeit erleichtern. TestHub stellt außerdem eine \gls{REST} \gls{API} innerhalb der Firma 
zur Verfügung, wodurch andere Programme die von TestHub verwendeten und verarbeiteten 
Informationen verwerten können. Durch die Verwendung von modernen und aufstrebenden Technologien,
wie Typescript oder der Programmiersprache Go, ist das Programm erweiterbar 
und zukunftssicher.\\

Jedoch gibt es bei \gls{JSS} nur wenige Entwickler, welche diese Sprachen beherrschen.
Allerdings werden auch keine komplexen sprachenspezifischen Konzepte genutzt, wodurch
der Einstieg in die Entwicklung dieses Projekts erleichtert wird. Zu diesem Zweck
wurde auch eine \textit{setup.md} erstellt, welche Entwicklern beim Start der
Entwicklung dieses Projekts helfen soll. Zusätzlich ist der Code umfangreich kommentiert.

\subsection{Ausblick}
Um die gesteigerte Effizienz auch wirklich belegen zu können sollten aussagekräftige
Studien zur Bedienbarkeit und Benutzung von TestHub durchgeführt werden.
Die nicht erfüllten Anforderungen geben einen kurzen Überblick über Möglichkeiten,
wie das Projekt weiterzuführen wäre. Besonders das Gantt-Diagramm würde eine gute Ergänzung
zur Übersichtlichkeit und Nachvollziehbarkeit der Teststruktur liefern und dafür
wäre auch noch genug Platz auf dem Dashboard.
Eine weitere Überlegung ist es, Benutzeraccounts einzuführen. Da bei einer Wartung
die Wartende Person manuell eingetragen wird, lässt dies Platz für Falscheingaben und Fehler. 
Durch die Einbindung von Benutzeraccounts kann man eher gewährleisten, dass die 
wartende Person auch tatsächlich die eingetragene Person ist. Dafür könnte OAuth
von Microsoft oder \gls{Jira} verwendet werden. OAuth bietet dabei die Möglichkeit, dass
die Verwaltung der Benutzerdaten nicht von TestHub übernommen werden muss. TestHub
wird lediglich als Programm registriert und leitet den Benutzer weiter, sodass er sich
bei dem entsprechendem Service anmeldet. Anschließend erhält TestHub die Accountdaten
des Nutzers. Beide dieser OAuth Möglichkeiten sind jedoch noch nicht für eigens entwickelte 
Anwendungen innerhalb der Firma nutzbar.

Da das Projekt als \gls{MVP} entworfen wurde, wurden derzeit keine Softwaretests
implementiert. Diese sind jedoch notwendig, um die korrekte Funktion des
Programms, auch nach weiteren Änderungen, zu gewährleisten. Besonderer Fokus sollte dabei
auf Tests der \gls{REST} \gls{API} gelegt werden, da diese für den Informationsgehalt 
zuständig ist. Bei \gls{API} Tests sollten neben Statuscode, Datenqualität auch 
falsche Benutzereingaben und damit Fehler getestet werden. Es ist auch Sinnvoll
Stresstests durchzuführen, um die Grenzen des Webservers zu kennen um diese nach 
so gut wie möglich zu vermeiden. Diese \gls{API} Test würden sich über Postman\footurl{https://www.postman.com}
durchführen lassen, da dort die Endpunkte schon dokumentiert sind.