\section{Einführung}

Die Effizienz in einem Unternehmen spielte schon immer mehr eine wichtige Rolle,
da es häufig viele Deadlines gibt, die eingehalten werden müssen. Effizientes 
Arbeiten benötigt jedoch nicht nur eine genaue Planung, sondern auch eine 
Möglichkeit für alle Beteiligten, die vorgesehenen Prozessschritte einzusehen 
und einzuhalten. Gerade im Bereich der Digitalisierung, gibt es viel Potential, 
Prozesse effizienter zu gestalten. Circa ein Drittel der Maschinenbau- und 
Anlagenbau Unternehmen sehen eine Effizienzsteigerung durch die 
Digitalisierung~\cite{Bre17}. So können Prozesse teilweise oder voll automatisiert
werden, um die Arbeitszeiten und den Arbeitsaufwand von Mitarbeitern und Mitarbeiterinnen
auf ein minimum zu beschränken. 

\subsection{Problemstellung}\label{sec:problems}
Bei Joyson Safety Systems in der Abteilung ``\gls{HOD} System Test'' werden die 
durchzuführenden Tests mittels der Ticketsoftware \gls{Jira} geplant. Jeder durchzuführende
Test erhält dabei pro Muster ein eigenes Ticket. Eine ganze Testreihe 
beinhaltet meistens hunderte einzelne Tests, welche in einer gewissen
Reihenfolge durchgeführt werden müssen. 

Nachdem eine Testreihe einplant wurde, wird jeden Morgen ein Tagesplan 
erstellt, in welchem genau beschrieben ist, welcher Test auf welche Weise 
durchzuführen ist. Dieser Plan liegt als Auszug eines \gls{Jira} Filters inf Form 
einer statischen HTML Datei vor (Siehe Abb.~\ref{fig:tagesplan}). Dadurch sind 
die Informationen, welche die \gls{Techniker} erhalten auf das Wichtigste begrenzt. 

Um die Zeit des Umbaus zu minimieren und die verfügbare Zeit einer Klimakammer
für Tests zu maximieren, muss ein \gls{Techniker} jedoch wissen, was er für den 
nächsten Test umbauen muss. Diese Information steht ausschließlich im \gls{Jira}  
Ticket zur Verfügung, weshalb das Nachsehen im Ticket für jeden Test einen
erheblichen Aufwand darstellt.

\begin{figure}[H]
    \includegraphics[width=\linewidth]{img/jira-ticket.png}
    \caption{\gls{Jira} Ticket}
\end{figure}

Solch ein \gls{Jira} Ticket ist zu unübersichtlich um schnell alle wichtigen 
Informationen zu erhalten. Wichtige Daten, wie etwa das zu verwendende Zubehör
sind nur in unstrukturierter Form in den Labels des Tickets vorhanden. 
Außerdem bringt das Heraussuchen jedes Tickets einen gewissen Zeitaufwand mit sich.\\

Ein weiteres Problem ist, dass die Wartungsinformationen momentan lediglich in einer Excel
Tabelle gespeichert und vom Wartungsbeauftragten gepflegt werden. Somit
weiß der \gls{Techniker} oder der \gls{Planer} nicht genau, ob das geplante 
Zubehör auch wirklich verwendet werden darf. Er muss sich folglich darauf verlassen, 
dass es rechtzeitig gewartet wurde, da ein Nachsehen in der Liste einen zu hohen
Aufwand bedeuten würde.\\

\pagebreak
Aus den zuvor genannten Situationen treten folgende Probleme hervor:

\begin{description}

    \item[1. Heraussuchen der Jiratickets für Informationen]\hfill \\
    Wenn der Techniker beispielsweise beim Umbau eines Tests wissen möchte,
    welcher Test folgt, muss er sich in \gls{Jira} anmelden, das Ticket über die
    Testrun-ID aus dem Tagesplan suchen, und im Ticket gucken, ob der nächste 
    Test verlinkt ist. Da in einer Klimakammer meistens 3 Muster getestet werden
    muss dieser Prozess für jedes weitere Muster wiederholt werden. Bei über 20 
    Klimakammern ist dies ein zu hoher Aufwand.
    

    \item[2. Wartungsdatum eines Zubehörs einsehen und anpassen]\hfill \\
    Wer wissen möchte, wann ein gewisses Zubehör gewartet werden soll, bzw. ob 
    es verwendet werden darf, muss in einer Excel Datei nach dem entsprechenden
    Zubehör suchen. Da diese Excel Datei im \gls{SVN} abgelegt ist, wird sie auch nur
    darüber mit allen weiteren Benutzern synchronisiert. Dabei kann es auch zu 
    unterschiedlichen Versionen der Datei kommen, wenn beispielsweise jemand 
    vor dem Einsehen der Datei seine lokale Working Copy nicht aktualisiert.

    \item[3. Universelle Schnittstelle]\hfill \\
    Da es Sinn macht, gerade die Wartungsdaten schon beim Vorbereiten der Tests
    zugänglich zu machen, damit der \gls{Techniker} schon vor dem Test weiß,
    ob alle verwendeten Ressourcen in Ordnung sind, ist es notwendig eine universell
    verwendbare Schnittstelle bereitzustellen. Dadurch ist es anderen Programmen 
    und Entwicklern möglich, die von TestHub bereitgestellten Informationen zu 
    erhalten und zu verarbeiten.
\end{description}


\subsection{Zielsetzung}
Das Programm wird unter dem Namen ``TestHub'' innerhalb der Firma \gls{JSS} 
veröffentlicht. Die Arbeit hat folgende Ziele:

\begin{itemize}
\item Entwicklung einer interaktiven Übersicht zum schnellen Überblick über Wartungstermine, aktuelle, vorherige und folgende Tests 
\item Aufbereitung und Visualisierung der unstrukturierten Daten in den Labels eines \gls{Jira} Tickets
\item Zentrale Speicherung, Einsicht und Anpassung von Testzubehör Informationen
\item Entwicklung einer offenen Schnittstelle, der zuvor genannten Punkte, zur Integration in andere Programme

\end{itemize}

Die Entwicklung von TestHub lässt sich in zwei grobe Punkte unterteilen:

\begin{description}

    \item[Entwicklung des Backends]\hfill \\
    Das Backend ist die zentrale serverseitige Software, welche die nötigen 
    Daten von \gls{Jira} oder einer Datenbank sammelt und aufbereitet über das Internet 
    versendet.
    

    \item[Entwicklung des UI]\hfill \\
    Um die Daten visuell und kompakt darzustellen, gibt es ein Interface, 
    welches die vom Backend empfangenen Daten übersichtlich auf einer Webseite
    anzeigt

\end{description}


Durch eine Trennung des Backends und des \gls{Frontend}s entstehen verschiedene 
Vorteile. Zum einen können andere Programme das Backend ebenfalls benutzen und
die nötigen Daten bei sich integrieren. Es lässt sich außerdem leicht erweitern.
Zum Anderen lässt sich das \gls{UI} leicht austauschen bzw. anpassen, da es lediglich 
die vorhandenen Daten anzeigt. \\

Testhub ist als \gls{MVP} entwickelt. Es werden daher nur die wichtigsten in 
Abschnitt Funktionen implementiert.


\subsection{Vorgehensweise}
Die vorliegende Arbeit wird ind mehrere Kapitel unterteilt:

\begin{description}

    \item[Kapitel 1: Einleitung:]
    In der Einleitung werden die Problemstellung und die Ziele mitsamt der Motivation
    der Arbeit beschrieben.
    
    \item[Kapitel 2: Grundlagen:]
    Im Kapitel über die Grundlagen werden Implementierungsmöglichkeiten diskutiert
    und abgewogen. Zusätzlich wird dort kurz die Abteilung und das zentrale Glied
    dieser Arbeit, die \gls{REST} \gls{API} erklärt.

    \item[Kapitel 3: Anforderungsanalyse:]
    Im dritten Kapitel werden die aktuellen Probleme analysiert und Anforderungen an 
    das zu entwickelnde Programm definiert.

    \item[Kapitel 4: Entwurf:]
    Im Entwurf wird sowohl die Softwarearchitektur des Programms als auch die Gestaltung 
    des \gls{UI} beschrieben.
    
    \item[Kapitel 5: Entwicklung:]
    Im Entwicklungsteil der Arbeit werden sowohl die Implementierung als auch 
    der Softwarelebenszyklus des gesamten Projekts beschrieben.

    \item[Kapitel 6: Validierung:]
    Die in Kapitel 3 ermittelten Anforderungen werden in diesem Kapitel dem 
    tatsächlichen Funktionen des Programms gegenübergestellt und verifiziert.

    \item[Kapitel 7: Fazit:]
    Im Fazit Kapitel werden die Ergebnisse der Arbeit zusammengefasst und weitere
    mögliche Schritte hinsichtlich der Weiterentwicklung besprochen.

\end{description}
\newpage



