
\section*{Anhang}\label{sec:anhang}
\addcontentsline{toc}{section}{Anhang} 

\appendix
\section*{A. Allgemeine Ergänzungen}
Dieser Arbeit wurde ein USB-Stick beigelegt, welcher den gesamten im Rahmen 
dieser Arbeit verfassten Quellcode und das zuvor angeführte Video enthält. Die
Ordnerstruktur ist wie folgt angelegt:

\begin{description}
    \item[\textit{./backend}] $\rightarrow$~die serverseitige Software (Go)

    \item[\textit{./data}] $\rightarrow$~Daten auf die der Server zur Laufzeit 
    zugreifen muss

    \item[\textit{./frontend}] $\rightarrow$~die clientseitige Software 
    (\textit{./frontend/ts}) und der Stylesheet (\textit{./frontend/css})

    \item[\textit{./static}] $\rightarrow$~statische Dateien wie Bilder oder Vektorgrafiken.
    Außerdem werden die von TypeScript kompilierten JavaScript Dateien und der 
    von TailwindCSS generierte Stylesheet hier abgelegt.

    \item[\textit{./templates}] $\rightarrow$~Die HTML Templates welche vom Server
    zu einem richtigen HTML Dokument zusammengesetzt werden.
    
    \item[\textit{./videos}] $\rightarrow$~Das zuvor angeführte Video.

\end{description}

\section*{B. Webservertest Quellcode}
\lstinputlisting[language=Python, caption=Python Flask Webserver]{resources/python.py}

\lstinputlisting[language=Go, caption=Go Webserver]{resources/go.go}

\lstinputlisting[language=Python, caption=Python Testskript]{resources/test.py}

\section*{C. Tagesplan und Ressourcenliste}
\begin{figure}[H]
    \includegraphics[width=\linewidth]{img/tagesplan_auszug_02_03_22.png}
    \caption{Auszug eines Tagesplans vom 02.03.2022}\label{fig:tagesplan}
\end{figure}

\begin{figure}[H]
    \includegraphics[width=\linewidth]{img/ressourcenliste_auszug_26_03_22.png}
    \caption{Übersichtsseite der Ressourcenliste vom 26.03.2022}\label{fig:ressourcenliste}
\end{figure}