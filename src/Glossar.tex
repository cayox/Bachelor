\newglossaryentry{HODg}{
    name = {HOD},
	description = {Hands On Detection; ein System das einem Lenkrad ermöglicht 
        Griffe zu erkennen}, 
}

\newglossaryentry{SVNg}{
	name = {SVN},
	description = {Apache Subversion; eine freie Software zur zentralen 
        Versionsverwaltung von Dateien und Verzeichnissen}
}

\newglossaryentry{MVPg}{
	name = {MVP},
	description = {Minimal Viable Product (deutsch: minimal funktionsfähiges
        Produkt); Implementierung von lediglich den essenziellen Funktionen eines Produkts~\cite{Rie11}},
}

\newglossaryentry{Techniker}{
    name = {Techniker},
    description = {Eine Person, die Tests aufbaut und durchführt}
}

\newglossaryentry{Planer}{
    name = {Planer},
    description = {Eine Person, die die Durchführung der Tests plant}
}

\newglossaryentry{Backend}{
    name = {Backend},
    description = {Das in einer Server-Client-Architektur, auf dem Server 
        ausgeführte Programm}
}

\newglossaryentry{Frontend}{
    name = {Frontend},
    description = {Das in einer Server-Client-Architektur, auf dem Client 
        ausgeführte Programm, welches das \gls{UI} beinhaltet}
}

\newglossaryentry{Open-Source}{
    name = {Open-Source},
    description = {Software, deren Quellcode frei zugänglich ist und die 
        beliebig kopiert, genutzt und verändert werden darf~\cite{Dud22b}}
}

\newglossaryentry{APIg}{
    name = {API},
    description = {Eine Schnittstelle, die von verschiedenen Programmen benutzt 
        werden kann um die zur Verfügung gestellten Funktionen zu nutzen}
}

\newglossaryentry{RESTg}{
    name = {REST},
    description = {ein Paradigma für die Softwarearchitektur von verteilten 
        Systemen, insbesondere für Webservices}
}

\newglossaryentry{objektorientiert}{
    name = {objektorientiert},
    description = {ein Programmierparadigma, mit dem die Konsistenz von 
        Datenobjekten gesichert werden kann und das die Wiederverwendbarkeit
        von Quellcode verbessert~\cite{ErKa20}}
}

\newglossaryentry{Framework}{
    name = {Framework},
    description = {Programmiergerüst mit einsatzbereitem Code oder Softwareplattform~\cite{Dud22c}}
}

\newglossaryentry{dynamisch typisiert}{
    name = {dynamisch typisiert},
    description = {Datentypen einer dynamisch typisierten Programmiersprache
        können sich zur Laufzeit ändern}
}

\newglossaryentry{statisch typisiert}{
    name = {statisch typisiert},
    description = {Datentypen einer statisch typisierten Programmiersprache
        werden zur Kompilierzeit festgelegt und sind nicht änderbar}
}

\newglossaryentry{Garbage Collector}{
    name = {Garbage Collector},
    description = {eine automatische Speicherverwaltung}
}

\newglossaryentry{JSONg}{
    name = {JSON},
    description = {ein Datenformat zum Austausch von Daten zwischen Anwendungen}
}

\newglossaryentry{URIg}{
    name = {URI},
    description = {ein Identifikator, welcher zur Bezeichnung von Webseiten etc. verwendet wird (z.B. \textit{www.bht-berlin.de})}
}

\newglossaryentry{HTMLg}{
    name = {HTML},
    description = {eine Sprache zum strukturellen Aufbau eines elektronischen Dokuments (z.B. eine Webseite)}
}

\newglossaryentry{Jira}{
    name = {Jira},
    description = {Jira (entwickelt von Atlassian) ist eine Webanwendung zur Fehlerverwaltung, Problembehandlung und zum operativen Projektmanagement}
}

\newglossaryentry{JQLg}{
    name = {JQL},
    description = {eine Sprache zum filtern der Jira Datenbank}
}

\newglossaryentry{CSSg}{
    name = {CSS},
    description = {eine Formatierungssprache zur visuellen Bearbeitung von HTML Dokumenten}
}

\newglossaryentry{DOMg}{
    name = {DOM},
    description = {ein Interface um die Struktur, den Inhalt oder das Aussehen einer Webseite anzupassen}
}

\newglossaryentry{threadsicher}{
    name = {threadsicher},
    description = {eine Softwarekomponente, welche das gleichzeitige Bearbeiten durch mehrere Programmteile ermöglicht, ohne dass diese sich gegenseitig behindern, nennt man threadsicher}
}