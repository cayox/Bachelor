\section{Grundlagen}
In diesem Kapitel werden die Grundlagen der im \gls{HOD} System Test erläutert.
Anschließend wird auf das zentrale Element der Arbeit, die REST API eingegangen.
Zusätzlich werden Programmiersprachen zur Entwicklung vom \gls{Backend} 
gegenübergestellt und eine potenzielle Verwendung abgewogen.

\subsection{HOD Systemtest}
Die Abteilung Hands-On-Detection entwickelt ein Lenkrad, welches durch eine 
kapazitive Matte im Lenkrad erkennen kann, ob und wie ein Lenkrad berührt wurde.
Dieses System muss nicht nur entwickelt sondern auch ausgiebig getestet werden.
Da Joyson nach dem V-Modell testet, werden die Tests auf mehrere Ebenen verteilt.
Die Ebene des Systemtests, testet das ganze System, das heißt es wird das ganze
Lenkrad mit der von Joyson entwickelten \gls{ECU} getestet.

\begin{figure}[H]
    \includegraphics[width=8cm,keepaspectratio]{img/duts_in_cc.jpg}
    \centering
    \caption{HOD Lenkräder in einer Klimakammer}
\end{figure}

Die verschiedenen Funktionen der Lenkräder wird bei -40°C bis 80°C getestet. 
Dabei werden meistens 3 Lenkräder in einer Klimakammer getestet. Alle, für 
diese Tests verwendeten Teile, wie Kabelbäume, Messhardware, die Klimakammer etc.,
werden Ressourcen genannt.


\subsection{REST API}\label{sec:restapi}
Seit der ersten Website von Tim Berners-Lee 1991 wächst das Web teilweise 
exponentiell. Allerdings war das Web nicht für solch ein rasantes Wachstum gemacht.
Es gab weder einheitliche Kommunikationsprotokolle, noch konnte die Infrastruktur
den Datenverkehr standhalten. Alarmiert von diesem Problem, entwickelte Roy Fielding
eine Web Architektur, welche die Bedingungen des Webs einheitlich lösen soll.
Diese Bedingungen lassen sich in 6 Kategorien zusammenfassen~\cite{Mas11}:

\begin{description}

    \item[1. Client-Server]\hfill \\
    Die Client-Server Struktur stellt eine Trennung der Anliegen dar. Der Client
    fordert dabei einen Dienst oder eine Information an, welche der Server erfüllt.
    Die Technologie welche zum Entwickeln von Client und Server verwendet wird,
    spielt dabei keine Rolle, solange sie das einheitliche Interface implementieren.

    \item[2. Einheitliches Interface]\hfill \\
    Mark Massé fasst in seinem Buch \textit{Rest Api: Design Rulebook}~\cite{Mas11}
    das einheitliche Interface in 4 Anforderungen zusammen: 
    
    \begin{enumerate}
        \item Identifikation von Ressourcen\hfill \\
        Jedes webbasierte Konzept (Ressource) muss über einen einzigartigen 
        \gls{URI} adressiert werden können. 

        \item Manipulation von Ressourcen durch Repräsentationen\hfill \\
        Clients müssen die Ressourcen manipulieren können, indem Sie mit verschiedenen
        Repräsentationen arbeiten. Beispielsweise kann ein Dokument sowohl in 
        \gls{JSON} für ein automatisiertes Programm als auch als \gls{HTML} für 
        einen Webbrowser repräsentiert werden. Dadurch lässt sich laut Massé eine
        Interaktion mit dem Dokument gewährleisten, ohne das Dokument und seinen
        Identifier zu verändern.

        \item Selbstbeschreibende Nachrichten\hfill \\
        Wenn der Client eine Anfrage schickt, ist dies nur der gewünschte Zustand
        der Ressource. Der tatsächlich aktuelle Zustand, ist repräsentativ in 
        der Antwort des Servers enthalten. Wenn also jemand einen Kommentar bei 
        YouTube schreibt, schlägt er nur den Inhalt des Kommentars vor. Ob der 
        Kommentar tatsächlich übernommen und angezeigt wird, hängt allein vom 
        Server ab. 
        Um diese selbstbeschreibenden Nachrichten mehr Informationen über die 
        versendete oder angefragte Ressource enthalten zu lassen, können Metadaten
        in den Nachrichten enthalten sein. Diese Metadaten können zum Beispiel
        die Art der Repräsentation, die Länge der Daten oder eine Authentifizierung 
        sein.\\

        Bei einer \gls{HTTP}-Nachricht werden diese Metadaten in die ``Header'' 
        geschrieben, welche vordefinierte Zwecke besitzen.

        \item Hypermedien als Antrieb des Applikationsstatuses\hfill \\
        Laut Massé, sind Links die ``Fäden die das Netz zusammennähen''~\cite[][4]{Mas11}.
        Daher sollte es in dem einheitlichen Interface die Möglichkeit eine Navigation
        der Informationen durch Links geben.

    \end{enumerate}

    \item[3. Schichtsystem]\hfill \\
    Durch ein Schichtsystem soll ermöglicht werden, Zwischensysteme, wie einen
    Proxy Server oder ein Gateway zu etablieren. Diese Systeme werden benötigt, 
    um beispielsweise Sicherheitsstandards zu erzwingen oder um viele gleichzeitige
    Anfragen auf die vorhandene Hardware zu balancieren (load balancing).

    \item[4. Caching]\hfill \\
    Caching ist das Zwischenspeichern von Informationen. Um den Server zu entlasten
    und somit Geld zu sparen und zusätzlich die Latenz des Clients zu verringern,
    sollten Ressourcen zwischengespeichert werden, sodass bei einer erneuten 
    Anfrage die zwischengespeicherte Version geladen wird und keine neue 
    Datenübertragung initialisiert werden muss. Caching wird von allen modernen 
    Webbrowsern automatisch betrieben, kann aber auch von den oben genannten
    Zwischensystemen umgesetzt werden.

    \item[5. Zustandslosigkeit]\hfill \\
    Die Anforderungen der Zustandslosigkeit beschreibt, dass alle kontextuellen 
    Informationen in der Anfrage des Clients enthalten sein muss, sodass der 
    Server keine Informationen zum Client speichern muss. Dadurch kann der Server
    wesentlich mehr Anfragen bearbeiten, das der Aufwand pro Anfrage gering 
    gehalten wird. Die Zustandslosigkeit trägt erheblich zur Skalierung der
    Architektur des Webs bei, laut Massé~\cite[][4]{Mas11}. 

    \item[6. Code-On-Demand]\hfill \\
    Code-On-Demand ist die Möglichkeit, ausführbare Skripte oder Programme vom 
    Server an den Client zu schicken. Da der Client jedoch den empfangenen Code
    verstehen und ausführen muss, ist dies die einzige optionale Anforderung an
    die Architektur des Webs.
\end{description}

Diese zuvor genannten Anforderungen, werden elegant durch eine \gls{REST} \gls{API}
erfüllt. REST steht für Representational State Transfer und beschreibt eine Ansammlung von
Regeln, nach welchem man seine \gls{API} architektonisch aufbauen sollte. Diese 
Regeln lassen sich jedoch auf die unterschiedlichsten Weisen implementieren. 

\subsection{HTTP Kommunikation}\label{sec:HTTP}
Das Hypertext Transfer Protokoll (kurz HTTP) regelt die Datenübertragung zwischen
Anwendungen. Viele \gls{REST} \gls{API}s verwenden HTTP, da dieses gewisse
Anforderung, wie die Zustandslosigkeit, bereits implementiert. Im Folgenden soll
die Funktionsweise vom HTTP erläutert werden, da dies grundlegend für sowohl ``TestHub''
als auch die \gls{JIRA} Kommunikation ist.

\subsubsection{Allgemeines}
HTTP implementiert verschiedene Anfragetypen und Headertypen. Die wichtigsten,
welche auch von der \gls{REST} \gls{API} von ``TestHub'' verwendet wird, werden 
hier aufgelistet:

\begin{table}[h!]
    \centering
    \begin{tabular}{|c | c|} 
     \hline
     \textbf{Anfragetyp} & \textbf{Beschreibung} \\ [1ex] 
     \hline
     GET & erhalten einer Ressource \\ [1ex]
     \hline
     POST* & erstellen einer Ressource oder Query-Abfrage \\ [1ex] 
     \hline
     PUT* & editieren einer Ressource \\ [1ex] 
     \hline
     DELETE & löschen einer Ressource \\ [1ex] 
     \hline
    \end{tabular}
    \caption{REST API HTTP Anfragetypen}
    * darf Daten im Body der Anfrage versenden
    \label{table:requests}
\end{table}

Außerdem hat jede Antwort des Servers auch einen entsprechenden Status Code, 
welcher verschiedene Aussagemöglichkeiten hat:

\begin{table}[h!]
    \centering
    \begin{tabular}{|c | c|} 
     \hline
     \textbf{HTTP Statuscode} & \textbf{Kategorie} \\ [1ex] 
     \hline
     1XX & Information \\ [1ex]
     \hline
     2XX & Erfolg \\ [1ex] 
     \hline
     3XX & Weiterleitung \\ [1ex] 
     \hline
     4XX & Client Error \\ [1ex] 
     \hline
     5XX & Server Error \\ [1ex] 
     \hline
    \end{tabular}
    \caption{HTTP Statuskategorien}
    \label{table:status}
\end{table}

Zusätzlich gibt es eine Vielzahl an Headern, welche verwendet werden können. 
``TestHub'' verwendet dabei nur die Standardheader wie \textit{Content-Length},
\textit{Date} und \textit{Content-Type}, um 
dem Client mitzuteilen, um welche Art der Repräsentation es sich handelt. 

\subsubsection{Kommunikation}
HTTP verwendet \gls{TCP}, welches für eine akkurate Übertragung der Daten sorgt.
Wenn der Client nun mit dem Server kommunizieren möchte, muss er zuerst eine \gls{TCP}
Verbindung aufbauen. 

\begin{figure}[H]
    \includegraphics[width=8cm,keepaspectratio]{img/Tcp-handshake.svg.png}
    \centering
    \caption{Diagramm des Beginns einer TCP-Verbindung~\cite{Wik10}}
\end{figure}

\makeatletter
\begin{description}
    
    \item[\textbf{SYN}]\hfill \\
    Der Client erstellt eine zufällige Zahlenfolge $x$ und versendet diese (und
    eventuell weitere TCP Optionen) in einem \textit{SYN} Packet an den Server.

    \item[\textbf{SYN-ACK}]\hfill \\
    Der Server inkrementiert $x$ um 1 und erstellt selbst eine zufällige 
    Zahlenfolge $y$. Der Server kann an dieser Stelle seine eigenen TCP Optionen
    zusammen mit den Zahlenfolgen $x$ und $y$ an den Client zurückschicken.

    \item[\textbf{ACK}\label{itm:ack}]\hfill \\
    Der Client inkrementiert sowohl $x$ als auch $y$ um 1 und sendet das Packet
    wieder an den Server zurück. 
\end{description}
\makeatother

Sobald der Server die \textit{\ref{itm:ack}} Nachricht empfangen hat, ist der Handshake
abgeschlossen. Von nun an können Daten versendet werden. Dies geschieht über
eine HTTP Nachricht, welche wie folgt aussehen kann:

\begin{lstlisting}[caption=Beispielhafte HTTP Anfrage]
    GET / HTTP/1.1
    Host: domain.com
    Accept-Language: de
\end{lstlisting}

In dieser Nachricht sind Anfragetyp, und die HTTP Version enthalten. \textit{Host}
und \textit{Accept-Language} sind dabei Header, welche gewisse Metadaten zur 
Anfrage enthalten. HTTP/1.1 ist dabei noch in Klartext gestaltet, und somit 
von Menschen lesbar. HTTP/2 verwendet das gleiche Prinzip, jedoch sind die 
Nachrichten als Frames verpackt und somit nicht mehr einfach so lesbar. \\

Sobald der Server alle vom Client angefragten Ressourcen gesammelt hat, 
sendet er eine Antwort:

\begin{lstlisting}[caption=Beispielhafte HTTP Antwort]
    HTTP/1.1 200 OK
    Date: Sat, 09 Oct 2010 14:28:02 GMT
    Content-Length: 29769
    Content-Type: text/html
    
    <!DOCTYPE html ... (29769 Bytes der angefragten Seite)
\end{lstlisting}

Üblicherweise sendet der Server zu seiner Antwort, für den Client wichtige Header,
wie der zuvor genannte \textit{Content-Type} Header, wodurch der Client weiß,
wie er die empfangenen Daten interpretieren soll. Die tatsächlichen Daten stehen
dabei ganz unten im sogenannten Body. Der Server sendet zusätzlich einen Status,
in diesem Fall ist das der Status 200, welcher für einen unspezifischen Erfolg
steht. \\

Schlussendlich kann die Verbindung geschlossen werden, um Platz für andere Anfragen
von anderen Clients zu schaffen oder die Verbindung kann für Folgeanfragen genutzt
werden~\cite{http02}.


\subsection{Programmiersprachen zur Backend Entwicklung}
Die Grundlage und demnach auch die Performance des Backends bildet die 
verwendete Programmiersprache. Joyson verwendet in der Firma hauptsächlich 
Python, C\# und Go. Da mit C\# nicht in der Abteilung 
\gls{HOD} System Test entwickelt wird, werden lediglich die 
Sprachen Python und Go gegenüber gestellt und ein Fazit zur verwendeten Sprache 
gebildet. \\

Sowohl Python als auch Go sind \gls{Open-Source} und somit frei verwendbar. Es 
gibt für beide Sprachen mehrere Bibliotheken zur Entwicklung einer \gls{REST} 
\gls{API}.

\subsubsection{Python}
Python wurde von Guido van Rossum 1989 in Amsterdam entwickelt. Es ist eine 
\gls{dynamisch typisiert}e Skriptsprache, welche auch \gls{objektorientiert} nutzbar 
ist. Python ist allerdings eine interpretierte Sprache, das bedeutet, das der 
compilierte Byte-Code in einer virtuellen Maschine ausgeführt wird. Diese 
virtuelle Maschine nennt man Interpreter~\cite{ErKa20}. \\

Trotz der umfangreichen Standardbibliothek von Python kann ein Webserver nicht
ohne Weiteres entwickelt werden. Hierfür wird ein \gls{Framework} benötigt, wie 
zum Beispiel Django, Flask oder FastAPI. Da Flask laut \textit{StackOverflow 
Developer Survey 2021} unter den genannten 3 Frameworks, das ist, welches am 
meisten genutzt wird~\cite{Sta21}, bezieht sich die folgende Gegenüberstellung 
auf diese Bibliothek. \\

\subsubsection{Go}
Die Programmiersprache Go wurde 2012 von der Firma Google veröffentlicht. Go 
wirbt damit, effizient und übersichtlich zu sein. Weiterhin kann in Go 
Concurrency, also das parallele ausführen von Code, sehr leicht umgesetzt werden.
Im Gegensatz zu Python ist Go \gls{statisch typisiert}, beide Sprachen besitzen jedoch
einen \gls{Garbage Collector}~\cite{Freeman2022}. \\

Go bietet in seiner Standardbibliothek schon das `net/http' package an, welches
verwendet werden kann um einen Webserver zu implementieren. 

\subsubsection{Gegenüberstellung}
Sowohl Python als auch Go sind bekannte Sprachen, wobei Python wesentlich 
beliebter ist. Python wird von ca. 48\% der an der \textit{StackOverflow 
Developer Survey 2021} teilgenommenen Entwickler verwendet, Go hingegen nur ca. 
10\%.Allerdings ist Go auf Platz 4 der Sprachen, die die meisten dieser Entwickler
lernen wollen~\cite{Sta21}. Demnach gibt es für beide Sprachen eine ausreichende
Community für Fragen oder Probleme. Außerdem werden beide Sprachen wahrscheinlich
in den nächsten Jahren weiterhin verwendet werden. \\

Da `TestHub' so schnell und responsive wie möglich sein sollte, wurde ein 
Geschwindigkeitstest durchgeführt. Es wurden jeweils ein Webserver in Go mit 
dem `net/http' Package und in Python mit Flask entwickelt. Beide Webserver stellen
einen \gls{REST} \gls{API} Endpunkt bereit, welcher eine Test Website~\cite{Bra22} verschickt.
Es wurden anschließend 100 asynchrone Request an diesen Endpunkt geschickt und die jeweilige
Zeit gemessen.

\begin{table}[h!]
    \centering
    \begin{tabular}{|c | c|} 
     \hline
     \textbf{Framework} & \textbf{$\diameter$ Antwortzeit [s]} \\ [0.5ex] 
     \hline
     Go (net/http) & 0,2242 \\ [0.5ex]
     \hline
     Python (Flask) & 2,2979 \\ [0.5ex] 
     \hline
     \textbf{Leistungsunterschied} & 10,2493 \\ [1ex] 
     \hline
    \end{tabular}
    \caption{$\diameter$ Antwortzeit zwischen Go und Python Webservers}
    \label{table:1}
\end{table}

\begin{figure}[H]
    \includegraphics[width=\linewidth]{img/results.png}
    \caption{Geschwindigkeitsvergleich eines Webservers: Python vs. Go}\label{fig:speedtestresults}
\end{figure}

\textit{Tests wurden lokal auf einem Dell Latitude 5590 (Intel Core i5-8350U CPU
@ 1.70GHz; 8GB RAM) durchgeführt, der Code lässt sich im Anhang finden} \\[2ex]

In Abbildung~\ref{fig:speedtestresults} ist die Antwortzeit in Sekunden über der Nummer 
der Anfrage dargestellt. Es ist schnell ersichtlich, dass Go ca. 10 mal schneller
als Flask ist, zumindest beim verschicken einer HTML Datei. Diese Ergebnisse 
decken sich mit den Web Framework Benchmarks von TechEmpower, wo Go beim 
versenden von Text auf Platz 18 vor Flask auf Platz 55 ist. Auch bei der 
\gls{JSON} Serialisierung liegt Go (Platz 22) weit vor Flask (Platz 59)
~\cite{Tec21}. Weiterhin ist die Performance des Go Servers mehr konsistent, als
die des Python Servers\\

\subsubsection{Fazit}
Da die Geschwindigkeit der Webseite und der \gls{REST} \gls{API} essentiell für 
die Benutzererfahrung ist, sowohl beim Laden der Seite, als auch beim bearbeiten
von \gls{HTTP} Anfragen, beispielsweise Suchanfragen, ist Go hier klar die
bessere Alternative. Des weiteren werden nur weniger Abhängigkeiten benötigt, da
Go schon viele benötigten Webserver Funktionen in seiner Standardbibliothek hat.
Somit ist das Programm, langfristig gesehen, weniger anfällig für Fehler bei 
Aktualisierungen der Bibliotheken.




